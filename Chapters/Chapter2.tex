\chapter{Алгоритм решения}
\section{Обзор алгоритмов}

Задача о назначениях имеет $(n!)^2$ возможных решений. Трехиндексная задача, в отличие от линейной, не может 
быть решена за линейное время и принадлежит к классу нп полных. Это было показано Karp[109 1] (book1).

Эйлер [75 1] (book1) изучал политроп (выпуклая оболочка всех возможных решений задачи (2)) трехиндексной аксиальной задачи о назначениях. Были открыты неравенства, разделяющие классы граней (class of facets), с помощью которых можно было получить оценки точного решения задачи (неточно). Независимо Balas и Saltzman [16 1] предложили $О(n^4)$ алгоритм,
который был улучшен Balas и Qi [15 1] до сложности $О(n^3)$

Эвристический алгоритм решение 3-АЗН, т.е. включающий практический метод, не являющийся гарантированно точным или оптимальным, но достаточный для решения поставленной задачи, был первым предложен Pierskalla [142 1]. Классическим точным алгоритмом решения является метод ветвей и границ. Большая часть алгоритмов делят задачу на 2 подзадачи, в каждой из которой одна переменная $x_{ijk}$ зафиксирована и равна 0 и 1 соотвественно, таким образом размерность подзадач уменьшается. Balas и Saltzman [17 1] разработали алгоритм, используя структуру задачи, может фиксисировать несколько переменных на одной ветви алгоритма. Burkard и Rudolf [44 1 ] поставили эксперементы с различными схемами решения 3-АЗН 

Hansen и Kaufman [100 1] описали метод одновременного решения прямой и двойственной задачи, который похож на Венгерский алгоритм для линейной задачи о назначениях. Идея метода заключалась в использовании гиперграфа вместо двудольного графа.

Другой подход, удобный для получения нижних оценок 3-АЗН -- представление (? функция, упорощение) Лагранжа. 
Внеся ограничения 3-АЗН в целевую функцию в виде множителей Лагранжа, можем получить представление задачи [*]

\[
L(\psi,\xi) = min \{ \sum^n_{i = 1} \sum^n_{j = 1}  \sum^n_{k = 1} 
(c_{ijk} + \psi_j + \xi_i) x_{ijk} - \sum^n_{j=1}\psi_j \sum^n_{i=1}\xi_j  \}
\]
где 
\begin{align*}
& \sum^n_{i = 1} \sum^n_{j = 1} x_{ijk} = 1, k=1,2, \ldots , n \\
& x_{ikj} \in {0,1}, \quad  1 \leq i,j,k \leq n \\
& \psi \in \mathbb{R}^n, \xi \in \mathbb{R}^n
\end{align*}

Так как $L(\psi, \xi)$ выпуклая функция, ее максимум может быть вычислен методом субградиентов. 
Frieze and Yadegar [83 1] описали вычислительный эксперемент, подтверждающие возможность применения подобного метода.
Balas and Saltzman [17] улучшили результаты, рассмотрев другой вид функции Лагранжа. 

В некоторых частных случаях возможно решение 3-АЗН. Например,  Burkard, Rudolf, and Woeginger [45] рассматривали задачи с разделяемыми весовыми коэфициентами, $c_{ijk}=u_iv_jw_k$, где $u_i, v_j, w_k$ -- некоторые неотрицательные числа. В своей работе они показали, что задача на максимум решается за полиномиальное время, тогда как задача на минимум остается нп полной.  Также за полиномиальное время решается задача на максимум, весовые коэфициенты которой берутся из массива Mogne  [42]
С друго стороны, Crama and Spieksma [61], рассматривая частные случаи 3-АЗН в терминах теории графов, обнаружили несколько вариантов 3-АЗН на минимум, решение для которых может быть получено за полиномиальное время. 

\section{Введение к алгоритму}

Обозначим через $f_{\mathrm{A}}$ и $f*$ приближенное (полученное при помощи алгоритма $\mathrm{A}$) 
и оптимальное значение целевой функции функции в некоторой конкретной задаче соответсвенно.

Будем говорить, что алгоритм  $\mathrm{A}$ имеет оценки $(\epsilon_{\mathrm{A}}, \delta_{\mathrm{A}})$
если выполнено неравентво 
\[
Pr \{ f_{\mathrm{A}} > (1+ \epsilon_{\mathrm{A}} f* )\} \leq \delta_{\mathrm{A}}
\],
где $\epsilon_{\mathrm{A}}$ есть оценка относительной погрешности решния, получаемого алгоритмом ${\mathrm{A}}$,
$\delta_{\mathrm{A}}$ -- вероятность несрабатывания алгоритма ${\mathrm{A}}$, что можно трактовать как долю случаев, когда алгоритм А не гарантирует точность в пределах $\epsilon_{\mathrm{A}}$.

Алгоритм А назвается асимптотически оптимальным если существую оценки $(\epsilon_{\mathrm{A}}, \delta_{\mathrm{A}})$
стремящиеся к нулю с ростом размерности. 

В дальнейшем будем полагать, что элементы матрицы весовых коэфициентов $c_{ijk}$ являются независимыми случайными величинами,  которые выбираются из $[a_n, b_n]$, где $a_n > 0$ и функции распределения одинаковы и определяются как

$
F_{\xi} (x) = Pr{\xi < x} 
$, где $\xi = (c_{ijk} - a_n ) / (b_n - a_n)$ -- нормализированная случайная переменная.

Через $M_n$ обозначим множество всех матриц, определенных выше. 

\section{Алгоритм}
Пусть $\phi = X \rightarrow \mathbb{N}$ -- любая целочислено значащая функция, при этом $1 < \phi_n < n$ 
\begin{enumerate}
\item Берем произвольную подстановку $\pi \in S_n$. Пусть $(d_{jk})$ - $n \times n$ 
матрица, содержащая элементы исходной матрицы $(c_{ijk})$, где индекс $j=\pi(i)$ такой, что
$$
d_{ij} = c_{\pi^{-1}(j)jk}
$$
для любых $1 \leq j$,$n \leq n$
Положим $f = 0 ; j =1 ; \mathrm{K}={1,2, \ldots , \phi_n}$. 
\item Выберем номер $\sigma(j)$ минимального элемента из множества $\mathrm{argmin} \, {d_{jk} | k \in K}$.
\item Полагаем $f = f + d_{j \sigma (j)} ; \mathrm{K} = \mathrm{K}  \setminus  {\sigma(j)} ; k=j+\phi_n$
\item Если $k \leq n $, то $K = K \bigcap {k}$.
\item $j = j + 1$
\item Повторяем п.2, пока j<n. В противном случае идем к п.7
\item Результатом работы алгоритма $\mathrm{A}(\phi_n)$ является значение функции $f$ целевой функции   
$f_{\mathrm{A}(\phi_n)}$. 
\end{enumerate}
\section{Блок-схема}
*пока уезжает в конец текста*

\begin{figure}[hp!]
  \includegraphics[width=\linewidth, height=\textheight,keepaspectratio]{Chapters/image/flowchart.png}
  \caption{A boat.}
  \label{fig:flowchart}
\end{figure}

\section{Комментарии к алгоритму}
Асимптотическое поведение трехиндексной аксиальной задачи о назначениях значительно отличается от поведения 
классической задачи о назначениях. В ряде статей Grundel Krohmal Oleveira было изучено поведение ожидаемого значения  оптимальной функции. Ими было доказано, что оно стремится к левой границе распределения весовых коэфициентов.

Теорема
При $b_n / a_n = o(n/ \mathrm{ln} n)$ алгоритм является ассимптотически оптимальным для 3-АЗН на классе матриц $М_n$
и его временная сложность  $O(n^2)$.

При $b_n / a_n = o(\mathrm{ln} n)$ алгоритм является ассимптотически оптимальным для 3-АЗН на классе матриц $М_n$
и его временная сложность  $O(n \mathrm{ln} n)$. 

Несмотря на то, что алгоритм показывает полиномиальное время выполения, возможен ряд улучшений
Например, заметно что алгоритм неустойчив относительно выбора начальной перестановки.
Также, 
\section{Вводимые модификации}
\subsection{Генерация нескольких начальных перестановок}
Изменим алгортим следующим образом

Пусть на начальном этапе дается не одна случайная перестановка, 
а $m$. Тогда на выходе можем выбрать лучшую ... Бред
\subsection{Выбор лучшей перестановки}
Введем функционал вида 

При известном точном решении будем говорить, что одна перестановка лучше другой
если она за тоже число шагов будет ближе сходится к точному решению 


\section{Итеративный алгоритм}
Добавим следующие шаги
После последнего шага сохраним результат, пойдем в начало и запустим алгоритм еще раз
Повторим M раз
Получим М выводов, в качестве ответа выберем устредненное значение. 