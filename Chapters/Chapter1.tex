\chapter{Введение}

Одной из фундаментальных задач комбинаторной оптимизации является задача о назначениях (ЗОН). 
В своей классической постановке эта задача звучит так: 

Имеется некоторое число работ и некоторое число исполнителей. Любой исполнитель может быть назначен на выполнение любой (но только одной) работы, но с неодинаковыми затратами. Нужно распределить работы так, чтобы выполнить работы с минимальными затратами.

Так как в данной форме рассматривается 2 множества -- работников $\mathrm{X}$ и работ $\mathrm{Y}$, затраты могут быть выражены ввиде $(c_ij) \in \mathrm{A}$, где $\mathrm{A}$ матрица из $Matr_{n \times n}$ и такая задача называется двухиндекской. 

В 1955 Куном был опубликовано решение этой задачи [link] в виде Венгерского алгоритма. В 1957 Манкрес определил, что 
алгорим является строго полиномианльным, а Карп улучшил его, добившись временной сложности $O(n^3)$

Естественно обобщить эту задачу, рассмотрев многоиндексную задачу о назначениях. Однако, уже 
для трехиндексной ЗОН было показано [кем?], что она принадлежит к классу нп-полных, т.е. не может быть решена за полиномиальное время. 