\chapter{Введение}

Одной из фундаментальных задач комбинаторной оптимизации является задача о назначениях (ЗОН). 
В своей классической постановке эта задача звучит так: 

Имеется некоторое число работ и некоторое число исполнителей. Любой исполнитель может быть назначен на выполнение любой (но только одной) работы, но с неодинаковыми затратами. Нужно распределить работы так, чтобы выполнить работы с минимальными затратами.

Так как в данной форме рассматривается 2 множества -- работников $\mathrm{X}$ и работ $\mathrm{Y}$, затраты могут быть выражены ввиде $(c_ij) \in \mathrm{A}$, где $\mathrm{A}$ матрица из $Matr_{n \times n}$ и такая задача называется двухиндекской. 

В 1955 Куном был опубликовано решение этой задачи [link] в виде Венгерского алгоритма. В 1957 Манкрес определил, что 
алгорим является строго полиномианльным, а Карп улучшил его, добившись временной сложности $O(n^3)$

Естественно обобщить эту задачу, рассмотрев многоиндексную задачу о назначениях. Однако, уже 
для трехиндексной ЗОН было показано [кем?], что она принадлежит к классу нп-полных, т.е. не может быть решена за полиномиальное время. 

Соответсвенно возникает проблема выбора достаточно хорошего решения. Само собой, эта задача, как и любая задача дискретной оптимизации, может быть решена полным перебором. Однако, слишком большая (экспоненциальная?) временная сложность для такого метода не позволяет использовать его в реальной жизни. Однако, имеет место улучшенная версия этого алгоритма -- метод ветвей и границ. В худшем случае он сводится к полному перебору, но чаще требует гораздо меньшего числа операций [ для получения \textit{приближенного} решения -- а не точный ли он?].

Б'ольшую практическую ценность представляют т.н. эвристические алгоритмы. Они за приемлимое время позволяют получить приближенное решение. Цель данной работы состоит в изучении одного из таких методов, для корого Гимади в [] было показано, что решения, полученные с помощью такого алгоритма сходятся при $n \rightarrow \inf$. Для достижения этих целей необходимо решить следующие задачи: 

\begin{itemize}  
\item Изучение математической модели 3-АЗОН
\item Изучить метод, предложенный Гимади
\item Программно реализовать этот метод
\item И провести его анализ
\end{itemize}
